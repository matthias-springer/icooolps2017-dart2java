%-----------------------------------------------------------------------------
%
%               Template for sigplanconf LaTeX Class
%
% Name:         sigplanconf-template.tex
%
% Purpose:      A template for sigplanconf.cls, which is a LaTeX 2e class
%               file for SIGPLAN conference proceedings.
%
% Guide:        Refer to "Author's Guide to the ACM SIGPLAN Class,"
%               sigplanconf-guide.pdf
%
% Author:       Paul C. Anagnostopoulos
%               Windfall Software
%               978 371-2316
%               paul@windfall.com
%
% Created:      15 February 2005
%
%-----------------------------------------------------------------------------


\documentclass[preprint]{sigplanconf}

% The following \documentclass options may be useful:

% preprint       Remove this option only once the paper is in final form.
%  9pt           Set paper in  9-point type (instead of default 10-point)
% 11pt           Set paper in 11-point type (instead of default 10-point).
% numbers        Produce numeric citations with natbib (instead of default author/year).
% authorversion  Prepare an author version, with appropriate copyright-space text.

\usepackage{amsmath}

\newcommand{\cL}{{\cal L}}

\begin{document}

\special{papersize=8.5in,11in}
\setlength{\pdfpageheight}{\paperheight}
\setlength{\pdfpagewidth}{\paperwidth}

\conferenceinfo{ICOOOLPS'17}{Month d--d, 20yy, City, ST, Country}
\copyrightyear{20yy}
\copyrightdata{978-1-nnnn-nnnn-n/yy/mm}\reprintprice{\$15.00}
\copyrightdoi{nnnnnnn.nnnnnnn}

% For compatibility with auto-generated ACM eRights management
% instructions, the following alternate commands are also supported.
%\CopyrightYear{2016}
%\conferenceinfo{CONF'yy,}{Month d--d, 20yy, City, ST, Country}
%\isbn{978-1-nnnn-nnnn-n/yy/mm}\acmPrice{\$15.00}
%\doi{http://dx.doi.org/10.1145/nnnnnnn.nnnnnnn}

% Uncomment the publication rights used.
%\setcopyright{acmcopyright}
\setcopyright{acmlicensed}  % default
%\setcopyright{rightsretained}

\titlebanner{Submitted for review to ICOOOLPS 2017}        % These are ignored unless
\preprintfooter{dart2java}   % 'preprint' option specified.

\title{{dart2java}: A Dart to Java Compiler}

\authorinfo{Andrew Krieger \and Stanislav Manilov \and Vijay Menon \\ \and Jennifer Messerly \and Leaf Peterson \and Matthias Springer}{Google Inc., Seattle}{akrieger@math.ucla.edu \and s.z.manilov@sms.ed.ac.uk \and vsm@google.com \\ \and jmesserly@google.com \and leafp@google.com \and matthias.springer@prg.is.titech.ac.jp}

\maketitle

\begin{abstract}
We present the design and implementation of \emph{dart2java}, an experimental Dart to Java compiler. It is implemented in Dart on top of the new \emph{Kernel} intermediate representation and currently supports many but not all Dart language constructs. dart2java is a playground to evaluate performance implications of running Dart code on the JVM and to investigate if it is possible to write Dart code in a largely Java-dominated environment.

This paper describes the architecture of dart2java, performance optimizations such as non-nullability of primitive types and generic specialization (and their implicaitons), as well as ideas for language interoperability, i.e., calling Java code from Dart and vice versa.
\end{abstract}

% 2012 ACM Computing Classification System (CSS) concepts
% Generate at 'http://dl.acm.org/ccs/ccs.cfm'.
\begin{CCSXML}
<ccs2012>
<concept>
<concept_id>10011007.10011006.10011008</concept_id>
<concept_desc>Software and its engineering~General programming languages</concept_desc>
<concept_significance>500</concept_significance>
</concept>
<concept>
<concept_id>10003752.10010124.10010138.10010143</concept_id>
<concept_desc>Theory of computation~Program analysis</concept_desc>
<concept_significance>300</concept_significance>
</concept>
</ccs2012>
\end{CCSXML}

\ccsdesc[500]{Software and its engineering~General programming languages}
\ccsdesc[300]{Theory of computation~Program analysis}
% end generated code

% Legacy 1998 ACM Computing Classification System categories are also
% supported, but not recommended.
%\category{CR-number}{subcategory}{third-level}[fourth-level]
%\category{D.3.0}{Programming Languages}{General}
%\category{F.3.2}{Logics and Meanings of Programs}{Semantics of Programming Languages}[Program analysis]

\keywords
keyword1, keyword2

\section{Introduction}
Very short description of Dart. Why dart2java?


\section{Architecture}
High-level overview of all compilation steps in a diagram (kernel AST, type inference, patching, code generation, for both SDK and user programs).


\section{Performance: Primitive Types Only}

\subsection{Primitive Types Only}

\subsection{Non-nullability}

\subsection{Generic Specialization}

\subsection{Benchmarks}


\section{Language Interoperability}

\subsection{Java Components in Dart}

\subsection{Dart Components in Java}


\section{Related Work}
Generic specialization in .NET using JIT, language interoperability in related projects


\section{Conclusion}
Preliminary findings, can the same techniques be applied to other projects?

% The 'abbrvnat' bibliography style is recommended.

\bibliographystyle{abbrvnat}

% The bibliography should be embedded for final submission.

\begin{thebibliography}{}
\softraggedright

\bibitem[Smith et~al.(2009)Smith, Jones]{smith02}
P. Q. Smith, and X. Y. Jones. ...reference text...

\end{thebibliography}


\end{document}
